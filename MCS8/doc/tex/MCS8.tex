\documentclass[10pt]{book}
\usepackage{amsmath}
\usepackage{amssymb}
\usepackage{latexsym}
\usepackage{ctex}
\usepackage{mathdots}
\usepackage{mathrsfs}
\usepackage{longtable}
\usepackage{supertabular}
\usepackage{multirow}
\usepackage{multicol}
\usepackage{array}
\usepackage{color, framed}
\usepackage{xcolor}
\usepackage{float}
\usepackage{listings}
\usepackage{threeparttable}
\usepackage{graphicx}
\usepackage{graphics}
\usepackage{enumerate}
\usepackage{footmisc}
\usepackage{comment}
\usepackage{tikz}
\usetikzlibrary{shapes.geometric}
\usepackage{lscape}
\usepackage[colorlinks]{hyperref}
\usepackage{geometry}
\setcounter{secnumdepth}{4}
\geometry{left=2.0cm, right=2.0cm, top=2.0cm, bottom=2.0cm}

\begin{document}

\lstset{numbers=left, 
	numberstyle= \tiny, 
	keywordstyle= \color{ blue!70},
	commentstyle=\color{red!50!green!50!blue!50},
	frame=shadowbox, 
	rulesepcolor= \color{ red!20!green!20!blue!20},
	xleftmargin=-1em,
	xrightmargin=-1em,
	tabsize=4,
	breaklines=true,
    basicstyle=\small
}

\definecolor{shadecolor}{rgb}{0.92, 0.92, 0.92}

\newcommand{\red}[1]{\textcolor[rgb]{1.0, 0.0, 0.0}{#1}}
\newcommand{\green}[1]{\textcolor[rgb]{0.0, 1.0, 0.0}{#1}}
\newcommand{\blue}[1]{\textcolor[rgb]{0.0, 0.0, 1.0}{#1}}
\newcommand{\greenblue}[1]{\textcolor[rgb]{0.0, 1.0, 1.0}{#1}}
\newcommand{\redB}[1]{\textcolor[rgb]{1.0, 0.0, 0.0}{\textbf{#1}}}
\newcommand{\greenB}[1]{\textcolor[rgb]{0.0, 1.0, 0.0}{\textbf{#1}}}
\newcommand{\blueB}[1]{\textcolor[rgb]{0.0, 0.0, 1.0}{\textbf{#1}}}
\newcommand{\greenblueB}[1]{\textcolor[rgb]{0.0, 1.0, 1.0}{\textbf{#1}}}
\newcommand{\mypath}[1]{/home/luoyin/Notes2016/ECG_Analysis/{#1}/}

\title{Intel MCS8系统在FPGA上的实现}
\author{罗胤}
\date{2018-03}
\maketitle

\tableofcontents

\chapter{CPU设计}
\section{命令法则}
\paragraph{模块约定}
\begin{itemize}
  \item 模块名均为小写
  \item 模块引脚均为大写, 输入引脚使用\blue{\_I}后缀, 输出引脚使用\blue{\_O}后缀, 无双向引脚
  \item 模块实例名以\blue{u}为前缀命名
\end{itemize}

\paragraph{信号约定}
\begin{itemize}
  \item 信号首字母小写, 第二字母大写, 其余字母按需求选择大小写
  \item 寄存器信号使用\blue{r}前缀, 线型信号使用\blue{w}前缀, 多位信号在前缀后附加\blue{s}
  \item 模块信号命令格式: 前缀-模块名-信号名
\end{itemize}

\section{模块组成}
\subsection{寄存器组}
\begin{itemize}
  \item 模块名: cpu\_regbank
  \item 总线接入: CPU内部wor总线
\end{itemize}

\section{信号通路}

%%%%%%%%%%%%%%%%%%%%%%%%%%%%%%
%%%%%%%%%%%%%%%%%%%%%%%%%%%%%%
\chapter{流水线设计}
\section{流水线分级}
采用七级流水线架构:
$F_{1}\rightarrow F_{2}\rightarrow F_{3}\rightarrow D\rightarrow E\rightarrow M/IO\rightarrow W$
\begin{itemize}
  \item $F_{1}$, $F_{2}$, $F_{3}$: 指令获取
  \item $D$: 译码准备
  \item $E$: 执行准备
  \item $M/IO$: RAM/IO准备
  \item $W$: 回写准备
\end{itemize}

\subsection{Stage D}
\subsubsection{寄存器变量}
\begin{itemize}
  \item valid
\end{itemize}

\subsection{Stage E}
\subsubsection{寄存器变量}
\begin{itemize}
  \item valid
  \item valA
  \item valS
\end{itemize}

\subsection{Stage M/IO}

\subsection{Stage W}

\section{流水线操作}
\subsection{Decode ($D\rightarrow E$)}
\subsubsection{操作内容}
\begin{itemize}
  \item 获取valA, valS
  \item 获取valH, valL
\end{itemize}

\subsection{Execute ($E\rightarrow M/IO$)}
\subsubsection{操作内容}
\begin{itemize}
  \item ALU算术操作:\\
  \textbf{IN}: valA, valS\\
  \textbf{OUT}: valE
  \item Mem/IO操作地址准备
\end{itemize}

\subsection{Memory/IO ($M/IO\rightarrow W$)}
\subsubsection{操作内容}
\begin{itemize}
  \item RAM Read/Write操作
  \item IO Read/Write操作
\end{itemize}

\subsection{WriteBack ($W\rightarrow F_{1}$)}
\subsubsection{操作内容}
\begin{itemize}
  \item 寄存器组更新
\end{itemize}

\section{流水线补充处理}
\subsection{Forward}
\subsubsection{概念}
\begin{itemize}
  \item 寄存器堆Forward: 寄存器堆相应地址上的值应该在执行完相应的指令后就进行修改, 但在流水线处理过程中, 仅在流水线的最后一级对寄存器堆进行写操作, 这会导致在Decode过程中可能取到未更新的寄存器堆的值, 从而引发执行错误, 故需要进行寄存器堆Forward.\\
  寄存器堆Forward与以下操作相关联:
  \begin{itemize}
    \item LRR操作: 与$valS(E/M/W)$相关
    \item ALU操作: 与$valE(M/W)$相关
    \item Mem操作: 与$valM(W)$相关
    \item IO操作: 与$valIO(W)$相关
  \end{itemize}
\end{itemize}
  
\subsection{Bubble}
\subsubsection{概念}
\paragraph{指令获取Bubble}
在两字节指令和三字节指令获取时, 暂停$F_{3}\rightarrow D$操作, 跳过一个或两个字节.

\paragraph{寄存器堆Forward Bubble}
在无法进行寄存器堆Forward操作时, 暂停$F_{1}\rightarrow F_{2}\rightarrow F_{3}\rightarrow D$操作, 等待寄存器堆Forward操作可以进行后恢复.



%%%%%%%%%%%%%%%%%%%%%%%%%%%%%%
%%%%%%%%%%%%%%%%%%%%%%%%%%%%%%
\chapter{as8008编译器设计}
\section{语法规则}

\section{需求分析}
\begin{itemize}
  \item 支持多个源文件
\end{itemize}

\section{设计思路}
\begin{itemize}
  \item 逐个扫描源文件, 解析指令
  \item 二次扫描源文件, 生成全局符号表, 并生成地址
  \item 三次扫描源文件, 生成编译后代码
\end{itemize}

%%%%%%%%%%%%%%%%%%%%%%%%%%%%%%
%%%%%%%%%%%%%%%%%%%%%%%%%%%%%%
\appendix
\chapter{Lattice MXO2片上ROM/RAM仿真代码}
\section{ROM仿真}
\subsection{ROM without output register}
\subsubsection{读操作}
\begin{itemize}
  \item 时钟$T_{n}$上沿, ROM Latch Address $A$, Data $D[A]$ output
  \item 时钟$T_{n+1}$上沿, Output Data $D[A]$ Stable
\end{itemize}

\subsection{ROM with output register}
\subsubsection{读操作}
\begin{itemize}
  \item 时钟$T_{n}$上沿, ROM Latch Address $A$, 
  \item 时钟$T_{n+1}$上沿, Latch Data $D[A]$ to output register
  \item 时钟$T_{n+2}$上沿, Output Data $D[A]$ Stable
\end{itemize}

\section{RAM仿真}

\end{document}
