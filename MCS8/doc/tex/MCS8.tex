\documentclass[10pt]{book}
\usepackage{amsmath}
\usepackage{amssymb}
\usepackage{latexsym}
\usepackage{ctex}
\usepackage{mathdots}
\usepackage{mathrsfs}
\usepackage{longtable}
\usepackage{supertabular}
\usepackage{multirow}
\usepackage{multicol}
\usepackage{array}
\usepackage{color, framed}
\usepackage{xcolor}
\usepackage{float}
\usepackage{listings}
\usepackage{threeparttable}
\usepackage{graphicx}
\usepackage{graphics}
\usepackage{enumerate}
\usepackage{footmisc}
\usepackage{comment}
\usepackage{tikz}
\usetikzlibrary{shapes.geometric}
\usepackage{lscape}
\usepackage[colorlinks]{hyperref}
\usepackage{geometry}
\setcounter{secnumdepth}{4}
\geometry{left=2.0cm, right=2.0cm, top=2.0cm, bottom=2.0cm}

\begin{document}

\lstset{numbers=left, 
	numberstyle= \tiny, 
	keywordstyle= \color{ blue!70},
	commentstyle=\color{red!50!green!50!blue!50},
	frame=shadowbox, 
	rulesepcolor= \color{ red!20!green!20!blue!20},
	xleftmargin=-1em,
	xrightmargin=-1em,
	tabsize=4,
	breaklines=true,
    basicstyle=\small
}

\definecolor{shadecolor}{rgb}{0.92, 0.92, 0.92}

\newcommand{\red}[1]{\textcolor[rgb]{1.0, 0.0, 0.0}{#1}}
\newcommand{\green}[1]{\textcolor[rgb]{0.0, 1.0, 0.0}{#1}}
\newcommand{\blue}[1]{\textcolor[rgb]{0.0, 0.0, 1.0}{#1}}
\newcommand{\greenblue}[1]{\textcolor[rgb]{0.0, 1.0, 1.0}{#1}}
\newcommand{\redB}[1]{\textcolor[rgb]{1.0, 0.0, 0.0}{\textbf{#1}}}
\newcommand{\greenB}[1]{\textcolor[rgb]{0.0, 1.0, 0.0}{\textbf{#1}}}
\newcommand{\blueB}[1]{\textcolor[rgb]{0.0, 0.0, 1.0}{\textbf{#1}}}
\newcommand{\greenblueB}[1]{\textcolor[rgb]{0.0, 1.0, 1.0}{\textbf{#1}}}
\newcommand{\mypath}[1]{/home/luoyin/Notes2016/ECG_Analysis/{#1}/}

\title{Intel MCS8系统在FPGA上的实现}
\author{罗胤}
\date{2018-03}
\maketitle

\tableofcontents

\chapter{CPU设计}
\section{命令法则}
\paragraph{模块约定}
\begin{itemize}
  \item 模块名均为小写
  \item 模块引脚均为大写, 输入引脚使用\blue{\_I}后缀, 输出引脚使用\blue{\_O}后缀, 无双向引脚
  \item 模块实例名以\blue{u}为前缀命名
\end{itemize}

\paragraph{信号约定}
\begin{itemize}
  \item 信号首字母小写, 第二字母大写, 其余字母按需求选择大小写
  \item 寄存器信号使用\blue{r}前缀, 线型信号使用\blue{w}前缀, 多位信号在前缀后附加\blue{s}
  \item 模块信号命令格式: 前缀-模块名-信号名
\end{itemize}

\section{模块组成}
\subsection{寄存器组}
\begin{itemize}
  \item 模块名: cpu\_regbank
  \item 总线接入: CPU内部wor总线
\end{itemize}

\section{信号通路}

%%%%%%%%%%%%%%%%%%%%%%%%%%%%%%
%%%%%%%%%%%%%%%%%%%%%%%%%%%%%%
\chapter{流水线设计}


%%%%%%%%%%%%%%%%%%%%%%%%%%%%%%
%%%%%%%%%%%%%%%%%%%%%%%%%%%%%%
\chapter{as8008编译器设计}
\section{语法规则}

\section{需求分析}
\begin{itemize}
  \item 支持多个源文件
\end{itemize}

\section{设计思路}
\begin{itemize}
  \item 逐个扫描源文件, 解析指令
  \item 二次扫描源文件, 生成全局符号表, 并生成地址
  \item 三次扫描源文件, 生成编译后代码
\end{itemize}


\end{document}
