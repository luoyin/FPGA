\documentclass[10pt]{book}
\usepackage{amsmath}
\usepackage{amssymb}
\usepackage{latexsym}
\usepackage{ctex}
\usepackage{mathdots}
\usepackage{mathrsfs}
\usepackage{longtable}
\usepackage{supertabular}
\usepackage{multirow}
\usepackage{multicol}
\usepackage{array}
\usepackage{color, framed}
\usepackage{xcolor}
\usepackage{float}
\usepackage{listings}
\usepackage{threeparttable}
\usepackage{graphicx}
\usepackage{graphics}
\usepackage{enumerate}
\usepackage{footmisc}
\usepackage{comment}
\usepackage{tikz}
\usetikzlibrary{shapes.geometric}
\usepackage{lscape}
\usepackage[colorlinks]{hyperref}
\usepackage{geometry}
\setcounter{secnumdepth}{4}
\geometry{left=2.0cm, right=2.0cm, top=2.0cm, bottom=2.0cm}

\begin{document}

\lstset{numbers=left, 
	numberstyle= \tiny, 
	keywordstyle= \color{ blue!70},
	commentstyle=\color{red!50!green!50!blue!50},
	frame=shadowbox, 
	rulesepcolor= \color{ red!20!green!20!blue!20},
	xleftmargin=-1em,
	xrightmargin=-1em,
	tabsize=4,
	breaklines=true,
    basicstyle=\small
}

\definecolor{shadecolor}{rgb}{0.92, 0.92, 0.92}

\newcommand{\red}[1]{\textcolor[rgb]{1.0, 0.0, 0.0}{#1}}
\newcommand{\green}[1]{\textcolor[rgb]{0.0, 1.0, 0.0}{#1}}
\newcommand{\blue}[1]{\textcolor[rgb]{0.0, 0.0, 1.0}{#1}}
\newcommand{\greenblue}[1]{\textcolor[rgb]{0.0, 1.0, 1.0}{#1}}
\newcommand{\redB}[1]{\textcolor[rgb]{1.0, 0.0, 0.0}{\textbf{#1}}}
\newcommand{\greenB}[1]{\textcolor[rgb]{0.0, 1.0, 0.0}{\textbf{#1}}}
\newcommand{\blueB}[1]{\textcolor[rgb]{0.0, 0.0, 1.0}{\textbf{#1}}}
\newcommand{\greenblueB}[1]{\textcolor[rgb]{0.0, 1.0, 1.0}{\textbf{#1}}}
\newcommand{\mypath}[1]{/home/luoyin/Notes2016/ECG_Analysis/{#1}/}

\title{FPGA}
\author{罗胤}
\date{2018-04}
\maketitle

\tableofcontents

\part{概述}
\chapter{总线概述}
\section{AMBA总线}
\subsection{概述}
\begin{itemize}
  \item AHB: 分为AHB Master和AHB Slave, 之间使用mux总线矩阵相连, Master选取使用Arbiter, Slave选取使用Decoder
  \item APB: 通过AHB/APB Bridge或ASB/APB Bridge与AHS或ASB相连, Bridge为Master, APB外设为Slave, APB外设使用或总线或mux总线相连.
\end{itemize}

\subsection{APB总线}

\section{Wishbone总线}


\section{ATA总线}

\end{document}
