\documentclass[10pt]{book}
\usepackage{amsmath}
\usepackage{amssymb}
\usepackage{latexsym}
\usepackage{ctex}
\usepackage{mathdots}
\usepackage{mathrsfs}
\usepackage{longtable}
\usepackage{supertabular}
\usepackage{multirow}
\usepackage{multicol}
\usepackage{array}
\usepackage{color, framed}
\usepackage{xcolor}
\usepackage{float}
\usepackage{listings}
\usepackage{threeparttable}
\usepackage{graphicx}
\usepackage{graphics}
\usepackage{enumerate}
\usepackage{footmisc}
\usepackage{comment}
\usepackage{tikz}
\usetikzlibrary{shapes.geometric}
\usepackage{lscape}
\usepackage[colorlinks]{hyperref}
\usepackage{geometry}
\setcounter{secnumdepth}{4}
\geometry{left=2.0cm, right=2.0cm, top=2.0cm, bottom=2.0cm}

\begin{document}

\lstset{numbers=left, 
	numberstyle= \tiny, 
	keywordstyle= \color{ blue!70},
	commentstyle=\color{red!50!green!50!blue!50},
	frame=shadowbox, 
	rulesepcolor= \color{ red!20!green!20!blue!20},
	xleftmargin=-1em,
	xrightmargin=-1em,
	tabsize=4,
	breaklines=true,
    basicstyle=\small
}

\definecolor{shadecolor}{rgb}{0.92, 0.92, 0.92}

\newcommand{\red}[1]{\textcolor[rgb]{1.0, 0.0, 0.0}{#1}}
\newcommand{\green}[1]{\textcolor[rgb]{0.0, 1.0, 0.0}{#1}}
\newcommand{\blue}[1]{\textcolor[rgb]{0.0, 0.0, 1.0}{#1}}
\newcommand{\greenblue}[1]{\textcolor[rgb]{0.0, 1.0, 1.0}{#1}}
\newcommand{\redB}[1]{\textcolor[rgb]{1.0, 0.0, 0.0}{\textbf{#1}}}
\newcommand{\greenB}[1]{\textcolor[rgb]{0.0, 1.0, 0.0}{\textbf{#1}}}
\newcommand{\blueB}[1]{\textcolor[rgb]{0.0, 0.0, 1.0}{\textbf{#1}}}
\newcommand{\greenblueB}[1]{\textcolor[rgb]{0.0, 1.0, 1.0}{\textbf{#1}}}
\newcommand{\mypath}[1]{/home/luoyin/Notes2016/ECG_Analysis/{#1}/}

\title{CPU-ARM}
\author{罗胤}
\date{2018-03}
\maketitle

\tableofcontents

\chapter{ARM指令集}
\section{指令类型}
ARM指令分类如下:
\begin{itemize}
  \item 数据处理指令 (Data processing). 以4位opcode标识, 最多16条数据处理指令.
  \item 数据加载与存储指令 (Load/Store)
  \begin{itemize}
    \item LDR    cond 01IPU0W1 Rn Rd addr\_mode
    \item LDRB   cond 01IPU1W1 Rn Rd addr\_mode
    \item LDRBT  cond 01I0U111 Rn Rd addr\_mode
    \item LDRD   cond 000PUIW0 Rn Rd addr\_mode
    \item LDREX  cond 00011001 Rn Rd SBO(4) 1001 SBO(4)
    \item LDRH   cond 000PUIW1 Rn Rd addr\_mode(4) 1011 addr\_mode
    \item LDRSB  cond 000OUIW1 Rn Rd addr\_mode(4) 1101 addr\_mode
    \item LDRSH  cond 000PUIW1 Rn Rd addr\_mode(4) 1111 addr\_mode
    \item LDRT   cond 01I0U011 Rn Rd addr\_mode
    \item STR    cond 01IPU0W0 Rn Rd addr\_mode
    \item STRB   cond 01IPU1W0 Rn Rd addr\_mode
    \item STRBT  cond 01I0U110 Rn Rd addr\_mode
    \item STRD   cond 000PUIW0 Rn Rd addr\_mode(4) 1111 addr\_mode
    \item STREX  
    \item STRH
    \item STRT
    \item 
  \end{itemize}
  \item 分支指令 (Branch)
  \begin{itemize}
    \item B, BL: cond 101L simm24
    \item BLX: 1111 101H simm24 / cond 00010010 SBO(12) 0011 Rm
    \item BX: cond 00010010 SBO(12) 0001 Rm
    \item BXJ: cond 00010010 SBO(12) 0010 Rm
  \end{itemize}
\end{itemize}

\subsection{Load/Store指令}
\subsubsection{Addressing Mode}
\subsubsection{Load and Store word or unsigned byte}
\begin{itemize}
  \item 指令格式: LDR|STR|\{<cond>\}\{B\}\{T\} Rd, <addressing\_mode>
  \item 指令编码: cond 01IPUBWL Rn Rd addr
  \begin{itemize}
    \item I, P, U, W, Rn, adddr: addressing mode
    \item L: Load (1) and Store (0)
    \item B: unsigned byte (1) and word (0)
  \end{itemize}
\end{itemize}

\subsubsection{Load and Store halfword or doubleword, and load signed byte}
\begin{itemize}
  \item 指令格式: LDR|STR|\{<cond>\}D|H|SH|SB Rd, <addressing\_mode>
  \item 指令编码: cond 000PUIWL Rn Rd addr(4) 1SH1 addr(4)
  \begin{itemize}
    \item I, P, U, W, Rn, adddr: addressing mode
    \item L, S, H
  \end{itemize}
\end{itemize}

\subsection{Data-processing指令}
\subsubsection{opcode1}
\begin{itemize}
  \item 指令格式: <opcode1>\{<cond>\}\{S\} <Rd>, <shifter\_operand>
  \item 指令编码: cond 00I opcode(4) S SBZ Rd shift(12)
  \item 指令: MOV (1101), MVN (1111)
\end{itemize}

\subsubsection{opcode2}
\begin{itemize}
  \item 指令格式: <opcode1>\{<cond>\} <Rn>, <shifter\_operand>
  \item 指令编码: cond 00I opcode(4) 1 Rn SBZ shift(12)
  \item 指令: CMP (1010), CMN (1011), TST (1000), TEQ (1001)
\end{itemize}

\subsubsection{opcode3}
\begin{itemize}
  \item 指令格式: <opcode1>\{<cond>\}\{S\} <Rd>, <Rn>, <shifter\_operand>
  \item 指令编码: cond 00I opcode(4) S Rn Rd shift(12)
  \item 指令: ADD (0100), SUB (0010), RSB (0011), ADC (0101), SBC (0110), RSC (0111), AND (0000), BIC (1110), EOR (0001), ORR (1100)
\end{itemize}

\subsection{Branch指令}
\subsubsection{B, BL}
\begin{itemize}
  \item 指令格式: B\{L\}\{cond\} <target\_address>
  \item 指令编码: cond 101L simm24
\end{itemize}

\subsubsection{BLX}
\begin{itemize}
  \item 指令格式: BLX <target\_address>
  \item 指令编码: 1111 101H simm24
  \item 指令格式: BLX\{<cond>\} <Rm>
  \item 指令编码: cond 00010010 SBO(12) 0011 Rm
  \item 指令格式: BX\{cond\} <target\_address>
  \item 指令编码: cond 00010010 SBO(12) 0001 Rm
  \item 指令格式: BXJ\{cond\} <target\_address>
  \item 指令编码: cond 00010010 SBO(12) 0010 Rm
\end{itemize}

\section{操作数类型}
\subsection{addressing\_mode 1}
\subsubsection{指令编码}
\begin{itemize}
  \item 编码区间: op[27..25], op[11..0]
  \item 编码类型: op[27..25]=000 (imm shift \& reg shift, by op[4])
  \begin{itemize}
    \item op[4]=0 (imm shift): shift\_imm(5) shift(2) 0 Rm
    \item op[4]=1 (reg shift): Rs 0 shift(2) 1 Rm
  \end{itemize}
  \item 编码类型: op[27..25]=001 (imm32)
  \begin{itemize}
    \item rotate\_imm imm8
  \end{itemize}
\end{itemize}

\subsubsection{具体指令}
\begin{itemize}
  \item \#<immediate>: rotate\_imm(4) imm8(8)
  \item <Rm>: SBZ(8) Rm
  \item <Rm>, LSL|LSR|ASR|ROR \#<shift\_imm>: shift\_imm(5) 00|01|10|11 0 Rm
  \item <Rm>, LSL|LSR|ASR|ROR <Rs>: Rs 0 00|01|10|11 1 Rm
  \item <Rm>, RRX: 0000 0110 Rm
\end{itemize}

\subsection{addressing\_mode 2}
\begin{itemize}
  \item imm offset/index: 010 PUBWL Rn Rd offset\_12
  \item reg offset/index: 011 PUBWL Rn Rd SBZ(8) Rm
  \item scaled reg offset/index: 011 PUBWL Rn Rd shift\_imm(5) shift(2) 0 Rm
  \item 选项位
  \begin{itemize}
    \item P (op[24]): post-indexed (P=0), offset or pre-indexed (P=1)
    \item U (op[23]): offset added to the base (U=1), subtracted from the base (U=0)
    \item B (op[22]): unsigned byte (B=1), word (B=0)
    \item W (op[21])
    \begin{itemize}
      \item P=0: LDR/LDRB/STR/STRB (W=0), LDRT/LDRBT/STRT/STRBT (W=1)
      \item P=1: base reg not updated (offset, W=0), base reg updated (pre-index, W=1)
    \end{itemize}
    \item L (op[20]): Load (L=1), Store (L=0)
  \end{itemize}
\end{itemize}

\chapter{ARM汇编语言}


\end{document}
